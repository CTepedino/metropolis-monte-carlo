\UseRawInputEncoding
\documentclass[12pt]{article}
\usepackage[utf8]{inputenc}
\usepackage[spanish, es-tabla]{babel}
\usepackage{float}
\usepackage{indentfirst}
\usepackage{colortbl}
\usepackage{wrapfig}
\usepackage{csvsimple}
\usepackage[normalem]{ulem}
\usepackage{caption}
\usepackage{subcaption}
\usepackage{vmargin}
\usepackage{multirow}
\usepackage{amsmath}
\usepackage{mathrsfs}
\usepackage{enumitem}
\usepackage{amsfonts}
\usepackage{hyperref}
\usepackage{hhline}
\usepackage{tikz}
\usepackage{xcolor}
\usepackage{listings}
\usepackage{graphicx} 
\usepackage{amsmath}
\usepackage{amsfonts}
\usepackage{amssymb}
\usepackage{fancyvrb}
\usepackage{times}

\newcommand{\keyword}[1]{\textcolor{blue}{\textbf{#1}}}


\setpapersize{A4}
\setmargins{2.5cm}      % margen izquierdo
{1.5cm}                 % margen superior
{16.5cm}                % anchura del texto
{23.42cm}               % altura del texto
{10pt}                  % altura de los encabezados
{1cm}                   % espacio entre el texto y los encabezados
{0pt}                   % altura del pie de página
{2cm}

\setlength{\parindent}{2em}
\setlength{\parskip}{1em}

\usepackage{titlesec}

\titleclass{\subsubsubsection}{straight}[\subsection]

\newcounter{subsubsubsection}[subsubsection]
\renewcommand\thesubsubsubsection{\thesubsubsection.\arabic{subsubsubsection}}
\renewcommand\theparagraph{\thesubsubsubsection.\arabic{paragraph}} % optional; useful if paragraphs are to be numbered

\titleformat{\subsubsubsection}
  {\normalfont\normalsize\bfseries}{\thesubsubsubsection}{1em}{}
\titlespacing*{\subsubsubsection}
{0pt}{3.25ex plus 1ex minus .2ex}{1.5ex plus .2ex}

\makeatletter
\renewcommand\paragraph{\@startsection{paragraph}{5}{\z@}%
  {3.25ex \@plus1ex \@minus.2ex}%
  {-1em}%
  {\normalfont\normalsize\bfseries}}
\renewcommand\subparagraph{\@startsection{subparagraph}{6}{\parindent}%
  {3.25ex \@plus1ex \@minus .2ex}%
  {-1em}%
  {\normalfont\normalsize\bfseries}}
\def\toclevel@subsubsubsection{4}
\def\toclevel@paragraph{5}
\def\toclevel@subparagraph{6}
\def\l@subsubsubsection{\@dottedtocline{4}{7em}{4em}}
\def\l@paragraph{\@dottedtocline{5}{10em}{5em}}
\def\l@subparagraph{\@dottedtocline{6}{14em}{6em}}
\makeatother

\setcounter{secnumdepth}{4}
\setcounter{tocdepth}{4}

\begin{document}

\begin{titlepage}
\begin{center}

{\includegraphics[width=0.4\textwidth]{itba-1.png}\par}
\vspace{1cm}
{\bfseries\LARGE Instituto Tecnológico de Buenos Aires \par}
\vspace{0.5cm}
{\scshape\Huge\underline {T.P. 2: Autómatas celulares} \par}
\vspace{0.4cm}
{\Large\itshape 72.25 - Simulación de Sistemas  \par}
\end{center}
\vfill

{\Large Integrantes \par}
\centering
{\Large Tepedino, Cristian - 62830 \par}
{\Large Bloise, Luca - 63004 \par}
{\Large Arias, Uriel Ángel - 63504 \par}

\vfill
{\Large\itshape Primer Cuatrimestre del 2025 \par}
\end{titlepage}

\setcounter{page}{0}
\tableofcontents
\clearpage

\section{Introducción}
En el siguiente informe se mostrarán los resultados de simular un modelo de Ising bidimensional para representar la opinión de un grupo de diferentes personas a través del tiempo en función de diferentes condiciones iniciales.

\section{Modelo de Ising}
El modelo de Ising es una red de spines que pueden apuntar hacia arriba o
hacia abajo. La red puede tener cualquier número de dimensiones y puede ser
de cualquier tipo (cuadrada, hexagonal, etc). En este caso, al ser cuadrada se cuenta con una grilla de dimensión $N$ donde cada spin viene a representar la opinión particular de una persona. De esta manera, dada una configuración inicial de opiniones y una probabilidad de cambio de opinión, se puede calcular el cambio de estado de cada individuo luego de verificar el estado de sus vecinos adyacentes. Para esto se sigue el siguiente algoritmo:

Considerando una Grilla cuadrada de N x N (Con Condiciones Periódicas de Contomo).
Cada Sitio representa un individuo con dos posibles $S_{ij}\pm1$

Evolución:

- Elegir un sitio (i. j) al azar.
- Calcular el signo de la suma de los 4 vecinos: $$sign(S_{i-1 j}\!+\!S_{i+1j}\!+\!S_{i j-1}\!+\!S_{i j+1})$$

    - Si es distinto de cero, entonces la opinión de la mayoria es:
$$O_{M}=sign(S_{i-1 j}\!+\!S_{i+1j}\!+\!S_{i j-1}\!+\!S_{i j+1})$$

    - Sino:
$$O_{M}=S_{ij}$$

- Con probabilidad (1-p) adoptar el estado de la mayoría: $$S\space^\prime_{ij}=O_{M}$$

- Con probabilidad p cambiar el estado del Sitio: $$S\space^\prime_{ij}=-S_{ij}$$


Este proceso se repite considerando un periodo elegido, que en particular para este trabajo fue el de un paso de Montercarlo que corresponde a $N^{2}$ actualizaciones individuales.

A partir de la recolección de los estados obtenidos, luego de una cantidad de pasos considerable, es posible tomar métricas que reflejen los cambios producidos, dentro de las cuales se encuentran el consenso y la susceptibilidad.

El consenso se define formalmente como:
    $$M(t)=|{\frac{1}{N^{2}}}\sum_{i,j}S_{i,j}(t)|$$

La susceptibilidad se define formalmente como:
    $$\chi=N^{2}\left(\langle M^{2}\rangle-\langle M\rangle^{2}\right)$$

\section{Implementación}
\subsection{Condiciones Iniciales}
Para poder realizar múltiples simulaciones con condiciones iniciales distintas se realizó una implementación parametrizada por medio de las propiedades del sistema que están disponibles en Java. De este modo, se pueden pasar los valores correspondientes a las siguientes propiedades: "n" (referido como $N$ en el modelo), "p", "mcSteps" (número de pasos de Montecarlo a completar), "seed" (valor utilizado para inicializar la clase que se encarga de generar números pseudoaleatorios para los pasos del proceso) y "output" (nombre del archivo de salida con el resultado de la simulación). Resulta relevante aclarar que solo para los casos de "n" y "p" se requiere algún valor especificado, el resto de las propiedades son opcionales dado que por defecto se correrá la simulación hasta que se envíe una señal para interrumpir al proceso, se usará una seed aleatoria desconocida y se escribirán los resultados de la simulación en un archivo de nombre "output.txt".

Considerando que se utilizó una red cuadrada representada por medio de una matriz, o de forma más rigurosa, un arreglo de arreglos, se decidió inicializar las posiciones iniciales de la misma con valores pseudoaleatorios. Para esto, valiéndose de la implementación provista por la clase Random de java.util, se obtiene un valor booleano que se traduce en alguno de los dos valores posibles para cada posición.

\subsection{Evolución del modelo}


\section{Simulaciones}

\section{Resultados}

\section{Conclusiones}

\end{document}4